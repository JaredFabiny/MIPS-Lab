\documentclass{article}
\usepackage{graphicx} % new way of doing eps files
\usepackage{listings} % nice code layout
\usepackage[usenames]{color} % color
\definecolor{listinggray}{gray}{0.9}
\definecolor{graphgray}{gray}{0.7}
\definecolor{ans}{rgb}{1,0,0}
\definecolor{blue}{rgb}{0,0,1}
% \Verilog{title}{label}{file}
\newcommand{\Verilog}[3]{
  \lstset{language=Verilog}
  \lstset{backgroundcolor=\color{listinggray},rulecolor=\color{blue}}
  \lstset{linewidth=\textwidth}
  \lstset{commentstyle=\textit, stringstyle=\upshape,showspaces=false}
  \lstset{frame=tb}
  \lstinputlisting[caption={#1},label={#2}]{#3}
}
\newcommand{\Line}[3]{
  	\lstset{backgroundcolor=\color{listinggray},rulecolor=\color{blue}}
  	\lstset{linewidth=\textwidth}
  	\lstset{commentstyle=\textit, stringstyle=\upshape,showspaces=false}
  	\lstset{frame=tb}
  	\lstinputlisting[label={#1}]{#2}
}


\author{your names}
\title{Lab title}

\begin{document}
\maketitle

\section{Introduction}
This lab allowed us to understand the usefulness and importance of GitHub, as well as Bash and why it is so useful for working on mass projects in teams. In this case, we will be using it extensively for all future labs.

\section{Preface}
Before getting into the commands that Bash can use in conjunction with GitHub, it is important to understand first what GitHub is and isn't.

At its absolute core, GitHub is a web site that helps programmers work on projects from a cloud level. It is filled with repositories that people can download and modify all their own, creating their own specialized versions and merge with other users who do the same thing, etc. It is not the same as most cloud-based file sharing; it is much more instant, and people don't all have to stick with the same set of files, instead opting to branch it into their own.

This report will provide the steps that were taken when getting started, in order of cloning the repository (or repo for short), setting user information, creating and adding a Vivado project to the repo, and pushing those changes to the remote server.

\section{Steps}
Below is the list of steps that must be taken

\Line{code:reg}{./LabGit/clone.txt}

This command allows you to "clone" the repository, with [url] replaced with the web address of the repository you want to clone from. This will pull the repository from the server and place it into the directory you are currently in for easy modification.

\Line{code:reg}{./LabGit/branch.txt}

This can be followed with a branch name of your choice to create a new branch of the repository. A branch will allow you to make your own modifications without pushing any of the changes to the master, only to your specific branch. Therefore, should you go too far from what was intended, you can always pull back from the master branch and start over, or in the worst case scenario if a file gets lost. These can also be followed by two modifiers; following it with "-a" will show a list of every branch in the repository, and "-vvv" will show the latest modifications to the branch.

\Line{code:reg}{./LabGit/checkout.txt}

If the branch is already created, this will allow you to switch branches as you wish. Nothing needs to be done other than this to bring all new/modified files into the mix; the command will do all the work for you.

\Line{code:reg}{./LabGit/add.txt}

When new files are created, the repository doesn't know they are part of the project until this command is executed. [file name(s)] can be replaced with the files proper, or an asterisk can be used to simply add all files with that name or similar to it (for instance, "a*" will add all files that start with the letter a).

\Line{code:reg}{./LabGit/config.txt}

These two lines allow the user to specify their name and email address, so that anyone who looks at the repository can easily know who made what changes and where to contact them.

\section{Implementation}
The Verilog code and explanations of why you implemented this way.  There are many ways to implement a given design in Verilog.  For instance why choose a case statement or ifs?  Why did you trigger on a negedge verses any signal change?  You should reference your code, for example, mine is in Listing~\ref{code:reg} on page~\pageref{code:reg}.  Note you don't have to reference the page, I just wanted to show you how you could, and the power of a label in \LaTeX\ .



\section{Test Bench Design}
This is where you discuss the test benches you wrote, and what they were designed to test.  You should discuss expected errors as well as random errors.  Be sure to include your Verilog code of the testbench, for example, mine is in Listing~\ref{code:regtest} on page~\pageref{code:regtest}.

\Verilog{Verilog code for testing a register.}{code:regtest}{../code/1_fetch/register.v}

\section{Simulation}
In this section you should show the results of your simulation, such as timing diagrams and explain any design issues you had to deal with.  A sample timing diagram is in Figure~\ref{fig:regtest} on page~\pageref{fig:regtest}.

\begin{figure}
\begin{center}
\caption{Timing diagram for the register test.}\label{fig:regtest}
\includegraphics[width=0.9\textwidth]{../images/registertiming.png}
\end{center}
\end{figure}

\section{Conclusions}
Overview the main points you want to stick in peoples minds and answer key questions you want to stick in peoples minds.  Did it work?  How well? What would you have done differently?  What did you learn?
\end{document} 