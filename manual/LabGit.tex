\documentclass{article}
\usepackage{graphicx} % new way of doing eps files
\usepackage{listings} % nice code layout
\usepackage[usenames]{color} % color
\definecolor{listinggray}{gray}{0.9}
\definecolor{graphgray}{gray}{0.7}
\definecolor{ans}{rgb}{1,0,0}
\definecolor{blue}{rgb}{0,0,1}
% \Verilog{title}{label}{file}
\newcommand{\Verilog}[3]{
  \lstset{language=Verilog}
  \lstset{backgroundcolor=\color{listinggray},rulecolor=\color{blue}}
  \lstset{linewidth=\textwidth}
  \lstset{commentstyle=\textit, stringstyle=\upshape,showspaces=false}
  \lstset{frame=tb}
  \lstinputlisting[caption={#1},label={#2}]{#3}
}
\newcommand{\Line}[3]{
  	\lstset{backgroundcolor=\color{listinggray},rulecolor=\color{blue}}
  	\lstset{linewidth=\textwidth}
  	\lstset{commentstyle=\textit, stringstyle=\upshape,showspaces=false}
  	\lstset{frame=tb}
  	\lstinputlisting[label={#1}]{#2}
}


\author{Jared Fabiny}
\title{Lab Git}

\begin{document}
\maketitle

\section{Introduction}
This lab allowed us to understand the usefulness and importance of GitHub, as well as Bash and why it is so useful for working on mass projects in teams. In this case, we will be using it extensively for all future labs.

\section{Preface}
Before getting into the commands that Bash can use in conjunction with GitHub, it is important to understand first what GitHub is and isn't.

At its absolute core, GitHub is a web site that helps programmers work on projects from a cloud level. It is filled with repositories that people can download and modify all their own, creating their own specialized versions and merge with other users who do the same thing, etc. It is not the same as most cloud-based file sharing; it is much more instant, and people don't all have to stick with the same set of files, instead opting to branch it into their own.

This report will provide the steps that were taken when getting started, in order of cloning the repository (or repo for short), setting user information, creating and adding a Vivado project to the repo, and pushing those changes to the remote server.

\section{Steps}
Below is the list of steps that must be taken

\Line{code:reg}{./LabGit/clone.txt}

This command allows you to "clone" the repository, with [url] replaced with the web address of the repository you want to clone from. This will pull the repository from the server and place it into the directory you are currently in for easy modification.

\Line{code:reg}{./LabGit/config.txt}

When working on a group project, it is important to set the user information. This will allow anyone looking at the code to know who changed something and where to contact them. This is important for cooperation.

\Line{code:reg}{./LabGit/add.txt}

Projects are not stagnant; new files are added all th time. For the sake of this report, we will focus on Vivado. During the prior work on the lab, two new files were created -- the Vivado project file, and the simulation. Because neither of these were originally part of the repo, if nothing is done before sending the project files back to the server, the new files will not be added. This can be hugely problematic if you try to clone the repo again; this command resolves that by allowing you to add the files. You can outright type the name and/or path of the file (depending on if you are in the correct directory or not), or you can use an asterisk modifier to add all files with a certain set of characters or all of them entirely. The files that are needed in this case are the XPR (the Vivado project file proper) and the WCFG file (the config file for a behavioral simulation).

\Line{code:reg}{./LabGit/push.txt}

At the end of a work session, this command can be useful for sending all of your work back to the server. It will upload all changed files to the repo branch with a report of what was changed, and your comment on the changes if you specified one. This will allow easy access to working on the files once again from any computer that has Internet access for easy editing.

\section{Conclusions}
GitHub is a very powerful tool. It allows for productivity unlike anything else when it comes to coding, and gives a detailed history of all changes and a history of it all, meaning no matter how far along you go, you can always look back to an earlier draft of your code to check on and see how you did everything. The usefulness of the tool really cannot be stressed enough, and the steps shown are only a handful of what Git is really capable of.
\end{document} 
